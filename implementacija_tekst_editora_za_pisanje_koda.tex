% Format teze zasnovan je na paketu memoir
% http://tug.ctan.org/macros/latex/contrib/memoir/memman.pdf ili
% http://texdoc.net/texmf-dist/doc/latex/memoir/memman.pdf
% 
% Prilikom zadavanja klase memoir, navedenim opcijama se podešava 
% veličina slova (12pt) i jednostrano štampanje (oneside).
% Ove parametre možete menjati samo ako pravite nezvanične verzije
% mastera za privatnu upotrebu (na primer, u b5 varijanti ima smisla 
% smanjiti 

\documentclass[12pt,oneside]{memoir}

% Paket koji definiše sve specifičnosti mastera Matematičkog fakulteta
\usepackage{matfmaster}
%
% Podrazumevano pismo je ćirilica.
%   Ako koristite pdflatex, a ne xetex, sav latinički tekst na srpskom jeziku
%   treba biti okružen sa \lat{...} ili \begin{latinica}...\end{latinica}.
%
% Opicija [latinica]:
%   ako želite da pišete latiniciom, dodajte opciju "latinica" tj.
%   prethodni paket uključite pomoću: \usepackage[latinica]{matfmaster}.
%   Ako koristite pdflatex, a ne xetex, sav ćirilički tekst treba biti
%   okružen sa \cir{...} ili \begin{cirilica}...\end{cirilica}.
%
% Opcija [biblatex]:
%   ako želite da koristite reference na više jezika i umesto paketa
%   bibtex da koristite BibLaTeX/Biber, dodajte opciju "biblatex" tj.
%   prethodni paket uključite pomoću: \usepackage[biblatex]{matfmaster}
%
% Opcija [b5paper]:
%   ako želite da napravite verziju teze u manjem (b5) formatu, navedite
%   opciju "b5paper", tj. prethodni paket uključite pomoću: 
%   \usepackage[b5paper]{matfmaster}. Tada ima smisla razmisliti o promeni
%   veličine slova (izmenom opcije 12pt na 11pt u \documentclass{memoir}).
%
% Naravno, opcije je moguće kombinovati.
% Npr. \usepackage[b5paper,biblatex]{matfmaster}

% Pomoćni paket koji generiše nasumičan tekst u kojem se javljaju sva slova
% azbuke (nema potrebe koristiti ovo u pravim disertacijama)
\usepackage{pangrami}

% Paket koji obezbeđuje ispravni prikaz ćiriličkih italik slova kada
% se koristi pdflatex. Zakomentarisati ako na sistemu koji koristite ovaj
% paket nije dostupan ili ako ne radi ispravno.
\usepackage{cmsrb}
\usepackage{minted2}

% Ostali paketi koji se koriste u dokumentu
%\usepackage{listings} % listing programskog koda

% Datoteka sa literaturom u BibTex tj. BibLaTeX/Biber formatu
\bib{implementacija_tekst_editora_za_pisanje_koda}

% Ime kandidata na srpskom jeziku (u odabranom pismu)
\autor{Бојан Барџић}
% Naslov teze na srpskom jeziku (u odabranom pismu)
\naslov{Имплементација текст едитора за писање кода}
% Godina u kojoj je teza predana komisiji
\godina{2024}
% Ime i afilijacija mentora (u odabranom pismu)
\mentor{др Весна \textsc{Маринковић}, доцент\\ Универзитет у Београду, Математички факултет}
% Ime i afilijacija prvog člana komisije (u odabranom pismu)
\komisijaA{др Милан \textsc{Банковић}, доцент \\ Универзитет у Београду, Математички факултет}
% Ime i afilijacija drugog člana komisije (u odabranom pismu)
\komisijaB{др Иван \textsc{Чукић}, доцент\\ Универзитет у Београду, Математички факултет}
% Ime i afilijacija trećeg člana komisije (opciono)
% \komisijaC{}
% Ime i afilijacija četvrtog člana komisije (opciono)
% \komisijaD{}
% Datum odbrane (obrisati ili iskomentarisati narednu liniju ako datum odbrane nije poznat)
\datumodbrane{15. јануар 2016.}

% Apstrakt na srpskom jeziku (u odabranom pismu)
\apstr{%
\pangrami
}

% Ključne reči na srpskom jeziku (u odabranom pismu)
\kljucnereci{анализа, геометрија, алгебра, логика, рачунарство, астрономија}

\begin{document}
% ==============================================================================
% Uvodni deo teze
\frontmatter
% ==============================================================================
% Naslovna strana
\naslovna
% Strana sa podacima o mentoru i članovima komisije
\komisija
% Strana sa posvetom (u odabranom pismu)
\posveta{Мами, тати и деди}
% Strana sa podacima o disertaciji na srpskom jeziku
\apstrakt
% Sadržaj teze
\tableofcontents*

% ==============================================================================
% Glavni deo teze
\mainmatter
% ==============================================================================

% ------------------------------------------------------------------------------
\chapter{Увод}
% ------------------------------------------------------------------------------

\section{Текст едитори}

\paragraph{}
Текст едитор је програм за измену текстуалних датотека. Најчешћи типови датотека
који се измењују коришћењем ових програма су једноставне текстуалне датотеке, 
датотеке које садржe изворни код, код језика за означавање као и конфигурационе датотеке. 
Неки од најпознатијих оваквих програма су \begin{latinica}\textit{Visual Studio Code}
\end{latinica} \cite{VSC}, \begin{latinica}\textit{Notepad}\end{latinica} \cite{Notepad},
\begin{latinica}\textit{Notepad++}\end{latinica} \cite{Notepad++}, \begin{latinica}\textit{VIM}
\end{latinica} \cite{VIM} и \begin{latinica}\textit{Emacs}\end{latinica} \cite{Emacs}.

\paragraph{}
Постоји више врста текст едитора. Постоје едитори једноставног текста, где информација
о датотеци предсавља само текст. Док такође постоје и едитори богатог текста, где
информација о датотеци поред текста садржи и неке додатне информације везано за изглед
текста (фонт, величина фонта, боја текста, маргине). Овај рад ће се бавити искључиво
едиторима једноставног текста.

\chapter{Структуре података у текст едиторима}
\paragraph{}
Текст у датотеци се може посматрати као линеарни низ карактера, 
тако се и свака измена над тим текстом може посматрати као додавање текста у
неки део низа или брисање подниза текста. Када би комплетан садржај текстуалне датотеке
која је отворена чувана као јединствен низ карактера, операције уметања и брисања
текста биле би временски јако скупе (\(O(n)\), где је \(n\) дужина текста) и њихово узастопно
извршавање над неким великим текстом би имало за последицу изузетну неефикасност текст едитора.
Како би се овај проблем превазишао осмишљене су различите структуре података које ове операције
врше ефикасније. Најпознатије овакве структуре су бафер са размацима 
(енг. \begin{latinica}\textit{gap buffer}\end{latinica}), уже (енг. 
\begin{latinica}\textit{rope}\end{latinica}) и табела делова (енг. 
\begin{latinica}\textit{piece table}\end{latinica}).

\section{Бафер са размацима}
\paragraph{}
Бафер са размацима (енг. \begin{latinica}\textit{gap buffer}\end{latinica}) је структура
података која се заснива на идеји да постоји један линеаран низ чија је средина
"празна", док се са леве и десне стране налази текст. Како се врше измене над неким делом 
текста тако се средина "помера" превлачењем последњег елемента леве стране на почетак десне
или обрнуто. Када се бафер попуни (размак није довољно велики за нову операцију), тада се 
алоцира нови бафер већих димензија, најчешће дупло већи, и у њега се копира текст из старог
бафера.
\paragraph{}
У односу на класичан низ карактера, бафер са размацима је доста 
ефикаснији јер не захтева реалокацију низа при свакој измени. Ефикасност операција уметања и 
брисања зависи највише од тога колики је размак између индекса на коме се  
врши измена текста и леве или десне границе размака. Ако је размак већ на потребној позицији, 
временска сложеност је \(O(1)\). Међутим, ако је размак на једном крају а потребно је да 
се мења други крај текста сложеност ће бити \(O(n)\). У просечном случају временска 
сложеност наведених операција је константна, јер су карактери најчешће писани један за другим. 
Треба напоменути и то да је с времена на време потребно вршити реалокацију низа, 
која је линеарне сложености.

\paragraph{}
Разматрана структура података је једноставна за имплементацију и често се користи за једноставне
текстуалне уносе. Познати текст едитор \begin{latinica}\textit{Emacs}\end{latinica} \cite{Emacs} 
користи ову структуру у својој имплементацији. На слици \ref{fig:gap_buffer} приказана је 
илустрација рада бафера.

\begin{figure}[!ht]
  \centering
  \includegraphics[width=1.1\textwidth]{images/Bafer_2.png}
  \caption{Илустрација бафера са размаком}
  \label{fig:gap_buffer}
\end{figure}

\paragraph{}
У првом кораку бафер је празан. У следећем кораку се додаје ниска \begin{latinica}\textit{"g"}\end{latinica} на почетну позицију. После тога на крај текста
се додаје ниска \begin{latinica}\textit{"ap\_"}\end{latinica}, а затим и ниска
\begin{latinica}\textit{"buffer"}\end{latinica}. У последњем кораку се додаје ниска
\begin{latinica}\textit{"between"}\end{latinica} на индекс број 6. Пошто је бафер  
у том тренутку пун, врши се реалокација. Када се текст прекопира у нови низ, 
помера се почетак празног дела на жељени индекс и додаје се нова ниска.

\section{Уже}
\paragraph{}
Уже (енг. \begin{latinica}\textit{rope}\end{latinica}) је бинарно стабло у коме сви чворови који нису
листови садрже број карактера у левом подстаблу тог чвора. У листовима се налазе ниске које 
садрже делове текста. Када се прође кроз листове од првог листа слева, добија се целокупан
текст.

\paragraph{}
Предност ове структуре у односу на обичну ниску је што операције као што су уметање, брисање
и претрага текста у просечном случају захтевају \(O(\log{}n)\) време (где је \(n\) дужина текста), 
а не \(O(n)\). Такође, није потребно чувати текст у непрекидном делу меморије, већ се чворови могу налазити на одвојеним местима.

\paragraph{}
Мане ужета јесу што је то комплексна структура и чешће су грешке како у раду са њом, тако и при
њеној имплементацији. Поред тога, заузима више меморије него обична ниска 
(због родитељских чворова који повезују листове).

\subsection{Претрага}
\paragraph{}
Приступ карактеру на \(i\)-том  индексу се врши тако што се рекурзивно пролази кроз стабло 
почевши од корена. Ако је индекс мањи од вредности у текућем чвору, настављамо претрагу у његовом левом 
подстаблу. У супротном од \(i\) се одузима вредност у тренутном чвору и наставља се претрага у
његовом десном подстаблу. Када се наиђе на лист враћа се \(i\)-ти карактер из ниске која се налази
у чвору. Сложеност операције је иста као код претраге бинарног стабла, а то је \(O(\log{}n)\).
Пример претраге дат је на слици \ref{fig:rope_search}.

\begin{figure}
  \centering
  \includegraphics[width=0.8\textwidth]{images/rope_search.png}
  \caption{Претрага помоћу ужета}
  \label{fig:rope_search}
\end{figure}

\paragraph{}
Тражи се карактер на индексу број 10. Прво се обилази корен који садржи дужину укупног текста, 
пошто је тражени индекс мањи од вредности у тренутном чвору иде се у његово лево подстабло. Следећи чвор
садржи вредност 9, што значи да је тражени индекс већи од вредности. Одузима се та вредност од траженог
индекса и претрага се наставља у десном подстаблу тренутног чвора. Сада је тражени индекс 1 и он се
поново упоређује са тренутним чвором. Пошто је 1 мање од 6 прелази се у лево подстабло. 
У следећем чвору вредност је 2 па се поново иде у лево подстабло. 
На крају се наилази на листаи у нисци коју он садржи се дохвата карактер на индексу 1 
и прослеђује се као повратна вредност.

\paragraph{}
Пре него што буде обрађена операција брисања, прво ће бити уведене две нове операције које ће
бити потребне за њену реализацију. Прва операција је надовезивање једног ужета на друго,
док је друга операција дељење једног ужета на два нова ужета по неком карактеру.


\subsection{Конкатенација}
\paragraph{}
Конкатенација или надовезивање је операција којом се на уже, које садржи текст \(t_1\), 
надовезује друго уже које садржи текст \(t_2\) и добија се ново уже које садржи текст \(t_1t_2\).
Конкатенација два ужета, \(r_1\) и \(r_2\), се врши тако што се направи нови родитељски чвор
чије ће лево дете бити корен од \(r_1\), а десно корен од  \(r_2\). Вредност новог чвора ће
бити сума дужина ниски у свим листовима ужета \(r_1\). Да би израчунали вредност новог корена
потребно је израчунати суму дужина ниски свих листова који се налазе у \(r_1\). То се постиже
рекурзивним обиласком \(r_1\), где се на коначну суму додаје вредност тренутног чвора и
затим се рекурзивно обилази десно подстабло тренутног чвора. Сложеност ове 
операције је \(O(\log{}n)\) за балансирано стабло. Пример конкатенације два 
ужета се може видети на слици \ref{fig:rope_concat}

\begin{figure}
  \centering
  \includegraphics[width=0.8\textwidth]{images/rope_concat.png}
  \caption{Конкатенација два ужета}
  \label{fig:rope_concat}
\end{figure}

\subsection{Дељење}
\paragraph{}
Дељење ниске \(s\) по индексу \(i\) на две ниске, \(s_1\) и \(s_2\),
где \(s_1\) садржи карактере од почетка ниске \(s\) до \(i\)-тог индекса (укључујући и карактер на 
\(i\)-том индексу), а \(s_2\) садржи карактере десно од \(i\)-тог индекса па до краја ниске,
се врши на следећи начин.

\paragraph{}
Најпре се налази \(i\)-ти карактер, ако је то последњи карактер ниске у листу онда се не ради ништа.
У супротном се дели лист на два нова листа, где је \(i\)-ти карактер последњи карактер ниске
левог листа. Нека је лист чији је \(i\)-ти карактер последњи у нисци означен са \(d\). Сада
се листови деле у две групе \(l_1\) и \(l_2\). У \(l_1\) се налазе листови лево од \(d\) као и сам \(d\), док се у \(l_2\) налазе листови десно од \(d\). Листови из се \(l_2\) одвајају од 
главног ужета и затим се спајају заједно. Алгоритам се завршава тако што се балансирају 
два новодобијена ужета.

\paragraph{} 
Сложеност је иста као код претходних операција. На слици \ref{fig:ropе_split} се може видети пример
дељења. Дељење се врши по индексу 10. Прво се проналази карактер на датом индексу на начин описан
у делу о претрази. Затим се проверава да ли је тај карактер последњи у листу, пошто јесте,
врши се подела на \(l_1\) и \(l_2\) по том листу.

\begin{figure}
  \centering
  \includegraphics[width=1.0\textwidth]{images/rope_split.png}
  \caption{Дељење ужета}
  \label{fig:ropе_split}
\end{figure} 

\subsection{Уметање}
\paragraph{}
Да би се нека ниска \(s\) уметнула y уже \(r\) на индексу \(i\), довољно је да се искористе претходно дефинисане операције дељења и конкатенације. Прво се уже \(r\) подели по индексу \(i\)
и добију се два ужета \(r_1\) и \(r_2\), затим се на \(r_1\) надовеже ниска \(s\) и добија се
ново уже \(r_3\). На крају се на уже \(r_3\) надовеже \(r_2\). Ова операција се састоји
од три операције сложености \(O(\log{}n)\), тако да је њена укупна сложеност \(O(\log{}n)\).

\subsection{Брисање}
\paragraph{}
Уколико је потребно обрисати сегмент ниске \(s\) смештене у ужету \(r\) која почиње на \(i\)-том
карактеру, а завршава се на \((i+l)\)-том карактеру, онда се то може урадити у три корака. 
Прво се изврши дељење ужета \(r\) по индексу \(i\) на два ужета \(r_1\) и \(r_2\), затим се \(r_2\)
подели по \(l\)-том индексу на \(r_3\) и \(r_4\). Последњи корак је надовезивање \(r_1\) и \(r_4\). 
Из истих разлога као код уметања, сложеност операције је \(O(\log{}n)\).

\section{Табела делова}
\paragraph{}
Табела делова (енг. \begin{latinica}\textit{piece table}\end{latinica}) је структура података
која се састоји од два бафера у којима се налази текст и повезане листе чворова који показују
на текст у баферу. Текст едитор који је имплементиран током рада на овој тези користи ову структуру
података, тако да ће бити детаљније описана него претходне две структуре података.

\paragraph{}
У први бафер се учитава текст који се већ налазио у датотеци коју смо
отворили и тај бафер се назива оригинални бафер (енг. \begin{latinica}\textit{original buffer}\end{latinica}). Од тренутка после учитавања текста из датотеке па надаље он остаје непромењен. 

\paragraph{}
У други бафер, који се назива бафер за додавање (енг. \begin{latinica}\textit{add buffer}\end{latinica}), 
се додаје текст који се током времена уписује у текст едитор. Сваки пут
када се додаје текст, без обзира где се додаје, он се додаје на његов крај. 
Важно је напоменути да када се брише текст, он се не брише из бафера за додавање, већ остаје у њему. 
На овај начин никада нема потребе за померањем текста који се већ налази унутар бафера.

\paragraph{}
Поставља се питање како је онда могуће исписати текст у правилном редоследу. То се постиже
коришћењем двоструко повезане листе чији су елементи тзв. дескриптори делова (енг. \begin{latinica}\textit{piece descriptors}\end{latinica}).
Сваки дескриптор садржи информацију о томе на који од
два бафера показује, на којем индексу бафера почиње текст и колика је дужина тог текста. 
Следећи код приказује чланске променљиве класе \begin{latinica}\textit{PieceDescriptor}\end{latinica} у имплементацији рада.

\begin{minted}{c++}
class PieceDescriptor {
// ...
private:
    SourceType m_source;
    size_t m_start;
    size_t m_length;
};
\end{minted}

\paragraph{}
Предност ове структуре је што је једноставним операцијама, уметањем и брисањем чворова из
повезане листе, могуће ефикасно вршити измене текста. Мана је потенцијално велико меморијско заузеће
бафера за додавање као и фрагментација на доста веома малих "делова", што чини претрагу
кроз листу мање ефикасном.

\paragraph{}
У наставку текста ће бити описане основе операције над овом структуром података. Најпре
ће бити уведене неке ознаке које ће бити употребљаване за све наведене операције:

\begin{itemize}
	\item \(n\) - дужина укупног текста,
	\item \(m\) - број елемената повезане листе,
	\item \(I_n \in \{0, 1,...,n-1\}\) - скуп валидних индекса текста,
	\item \(I_m \in \{0, 1,...,m-1\}\) - скуп валидних индекса повезане листе,
	\item \(p_i\) - почетак \(i\)-тог дескриптора, где је \(i \in I_m\),
	\item \(d_i\) - дужина \(i\)-тог дескриптора, где је \(i \in I_m\).
\end{itemize}

\subsection{Уношење текста}
\paragraph{}
Ако се додаје неки текст дужине \(d\) у текући текст почев од позиције
\(i \in \{0,1,...,n\}\), прво ће бити размотрена два специјалана случаја за индекс \(i\):

\begin{enumerate}
	\item \(i=0\): У овом случају се додаје нови дескриптор на почетак листе.
	\item \(i=n\): У овом случају се додаје нови дексриптор на крај листе.
\end{enumerate}

За све остале случајеве пролази се кроз листу слева на десно. Нека је сума дужина свих дескриптора
пре \(ј\)-тог, при чему је \(j \in I_m\), једнака \(s_j\). Заустављамо се 
када буде важио услов \(s_j + d_j \geq i\), где је \(j\) индекс елемента листе на ком
се налазимо. Разликују се два случаја:

\begin{enumerate}
	\item \(s_j + d_j = i\): У овом случају убацује се нови дескриптор испред \(j\)-тог
	елемента повезане листе дескриптора.
	\item \(s_j + d_j > i\): У овом случају се дели тренутни дескриптор на два тако
	да леви садржи \(i - s_j\) почетних карактера оригиналног дескриптора, а десни остале.
	Ово се постиже тако што се дужина \(j\)-тог дескриптора поставља на \(i - s_j\), затим
	се прави нови дескриптор чији је почетак \(p_j + (i - s_j)\), а дужина \(d_j - (i - s_j)\)
	и умеће се испред \(j\)-тог. На крају се додаје дескриптор са новим текстом између ова
	два.
	
\end{enumerate}

\paragraph{}
Сложеност ове операције је \(O(m + d)\), јер се кроз листу пролази највише \(m\) пута.
Додавање новог дескриптора и дељење претходног на два су операције сложености \(O(1)\), док
додавање новог текста дужине \(d\) у бафер има сложеност \(O(d)\).

\subsection{Брисање}
\paragraph{}
Брисање неког дела текста чији се индекси налазе у целобројном интервалу \([p, k)\), где су 
\(p \in I_n\), \(k \in I_n \cup \{n\}\), се врши на аналоган начин као код додавања. 
Пролази се кроз повезану листу све док не буде важио услов \(s_i + d_i \geq p\) за неко 
\(i \in I_m\). Затим се редом пролази кроз све дескрипторе који садрже текст чији су индекси из опсега
 \([p_j, p_j+d_j)\) за \(ј \geq i\), \(j \in I_m\) и  имају пресек са \([p, k)\) и ажурирамо 
 их по следећим правилима:

\begin{enumerate}
	\item Ако је пресек суфикс опсега \([p_j, p_j+d_j)\) дужине \(d_p\), онда се одузима суфикс 
	од дескриптора тако што му се смањује дужина за \(d_p\).
	
	\item Ako je пресек префикс опсега \([p_j, p_j+d_j)\) дужине \(d_p\), онда се одузима префикс
	 од дескриптора тако што му се помера почетак у десно за \(d_p\) и смањује дужина за исто
	 толико.
	 
	\item Ако је пресек цео опсег \([p_j, p_j+d_j)\), онда се брише цео дескриптор.
	 
	\item Ако је \([p, k)\) садржан у \([p_j, p_j+d_j)\) и није ни префикс ни суфикс опсега,
	онда се дескриптор дели на два нова, где први садржи опсег \([p_j, p)\), а други 
	\([k, p_j+d_j)\).
\end{enumerate}
\paragraph{}
Треба приметити да случај 1. може важити само за последњи промењени дескриптор, као и
да случај 2. може важити само за први промењени. Случај 3. може важити за све. Ако за
неки дескриптор важи случај 4, онда је он једини промењени дескриптор.

\paragraph{}
Временска сложеност је \(O(m)\), јер се пролази највише \(m\) пута кроз листу и свако
ажурирање дескриптора је сложености \(O(1)\).

\subsection{Исписивање}
\paragraph{}
Уколико је потребно да се испише целокупан текст на стандардни излаз или у неку датотеку,
то се постиже помоћу следеће процедуре. Пролази се кроз листу дескриптора од почетка до краја
и за сваки дескриптор се исписује подниска одговарајућег бафера која починње на индексу 
\(p_j\) и има дужину \(d_j\), где је \(j \in I_m\). На слици \ref{fig:piece_table} се може
видети како изгледа приказ једног текста помоћу табеле делова.

\begin{figure}[!ht]
	\centering
	\includegraphics[width=1.0\textwidth]{images/piece_table.png}
	\caption{Приказ рада табеле делова}
	\label{fig:piece_table}
\end{figure}

\paragraph{}
По овом механизму где се целокупан текст добија тако што надовезујемо "делове" бафера је сама
структура података добила име. Један од познатијих текст едитора који од 2018. 
користи табелу делова у својој имплементацији је \begin{latinica}\textit{Visual Studio Code}\end{latinica} \cite{VSC}.

\paragraph{}
Сложеност ове операције је \(O(\sum_{i=0}^{m-1} d_i)\) тј једнака је суми
дужина свих дескриптора. Пошто је сума свих дужина једнака дужини укупног текста, онда
се сложеност може једноставније записати као \(O(n)\).

% ------------------------------------------------------------------------------
\chapter{Разрада}
\label{chp:razrada}
% ------------------------------------------------------------------------------

\pangrami

\pangrami

% ------------------------------------------------------------------------------
\chapter{Закључак}
% ------------------------------------------------------------------------------
\pangrami

\pangrami

% ------------------------------------------------------------------------------
% Literatura
% ------------------------------------------------------------------------------
\literatura

% ==============================================================================
% Završni deo teze i prilozi
\backmatter
% ==============================================================================

% ------------------------------------------------------------------------------
% Biografija kandidata
\begin{biografija}
\textbf{Вук Стефановић Караџић} (\emph{Тршић, 26. октобар/6. новембар
  1787. — Беч, 7. фебруар 1864.}) био је српски филолог, реформатор
српског језика, сакупљач народних умотворина и писац првог речника
српског језика.  Вук је најзначајнија личност српске књижевности прве
половине XIX века. Стекао је и неколико почасних доктората.
Учествовао је у Првом српском устанку као писар и чиновник у
Неготинској крајини, а након слома устанка преселио се у Беч,
1813. године. Ту је упознао Јернеја Копитара, цензора словенских
књига, на чији је подстицај кренуо у прикупљање српских народних
песама, реформу ћирилице и борбу за увођење народног језика у српску
књижевност. Вуковим реформама у српски језик је уведен фонетски
правопис, а српски језик је потиснуо славеносрпски језик који је у то
време био језик образованих људи. Тако се као најважније године Вукове
реформе истичу 1818., 1836., 1839., 1847. и 1852.
\end{biografija}
% ------------------------------------------------------------------------------

\end{document} 